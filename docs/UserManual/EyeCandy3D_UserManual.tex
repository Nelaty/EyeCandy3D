\documentclass[12p, paper=a4, leqno, colorinlistoftodos]{article}
\usepackage{wrapfig}
\usepackage[utf8]{inputenc}
\usepackage[ngerman]{babel}
\usepackage{amsmath}
\usepackage{amsfonts} 
\usepackage{amssymb}
\usepackage{enumerate}
\usepackage{graphicx}
\usepackage{wrapfig}

\usepackage[parfill]{parskip}
\usepackage{subcaption}
\usepackage{fix-cm}
\usepackage{transparent}
\usepackage{color}
\usepackage{xcolor}
\usepackage{tocloft, url}
\usepackage{hyperref}
\usepackage{etoolbox}
\usepackage{fancyhdr}
\usepackage[german]{fancyref}
\usepackage{lipsum}	
\usepackage{sectsty}
\usepackage{pdfpages}
\usepackage{tcolorbox}
\usepackage{todonotes}


\usepackage{enumitem}
\usepackage[bottom]{footmisc}

\usepackage{forest}
\usepackage{float}
\usepackage{longtable}
\usepackage[normalem]{ulem}
\usepackage{multirow}
\usepackage[font=small,labelfont=bf,tableposition=top]{caption}
\usepackage{tikz}
\usepackage{calc}
\usepackage{geometry}
\usepackage{listings,lstautogobble}
\usepackage{algorithm}
\usepackage{algorithmicx}
\usepackage[noend]{algpseudocode}

\usepackage{array}
\usepackage{booktabs}
\usepackage{xcolor,colortbl}
\usepackage{setspace}
\usepackage{titlesec}
\usepackage[nottoc,notlot,notlof,numbib]{tocbibind}

\usepackage{cite}


\hypersetup{
	colorlinks,
	linkcolor={black},
	citecolor={blue!50!black},
	urlcolor={blue!80!black}	
}

% BEGIN COLORS
\definecolor{CBlue}{RGB}{1,0,119}
\definecolor{CBlack}{rgb}{0.0,0.0,0.0}
\definecolor{CLightGrey}{rgb}{0.664,0.664,0.664}
\definecolor{LightOrange}{rgb}{1.0,0.9,0.63}
\definecolor{DarkOrange}{rgb}{1.0,0.8,0.22}
\definecolor{LightCyan}{rgb}{1.0,0.9,0.63}
\definecolor{CWhite}{RGB}{0,0,0}
\definecolor{CLink}{rgb}{0.5,0.0,0.0}

\newcommand{\CTitle}{CBlue}
% BEGIN Table row/column colors{102,217,239}
\definecolor{CEven0}{RGB}{208,238,244}
\definecolor{CUneven0}{RGB}{255,255,255}
\definecolor{CEven1}{rgb}{1.0,1.0,1.0}
\definecolor{CUneven1}{rgb}{1.0,1.0,1.0}
% END Table row/column colors
% END COLORS


% BEGIN NEWENVIRONMENTS
\newenvironment{packed_itemize}
{\begin{itemize}
		\setlength{\itemsep}{0pt}
		\setlength{\parskip}{0pt}
		\setlength{\parsep}{0pt}
	}{\end{itemize}}

\newenvironment{packed_description}
{\begin{description}
		\setlength{\itemsep}{0pt}
		\setlength{\parskip}{0pt}
		\setlength{\parsep}{0pt}}
	{\end{description}}

\newenvironment{packed_enumerate}
{\begin{enumerate}
		\setlength{\itemsep}{0pt}
		\setlength{\parskip}{0pt}
		\setlength{\parsep}{0pt}
	}{\end{enumerate}}

% END NEWENVIRONMENTS


% BEGIN NEWCOMMANDS
% tables
\newcommand{\tabularPaddingSmall}{\renewcommand{\arraystretch}{0.75}}
\newcommand{\tabularPaddingDefault}{\renewcommand{\arraystretch}{1}}
\newcommand{\tabularPaddinglarge}{\renewcommand{\arraystretch}{1.25}}
\newcommand{\tabularPaddingLarge}{\renewcommand{\arraystretch}{1.5}}
\newcommand{\tabularPaddingLARGE}{\renewcommand{\arraystretch}{2}}
\newcommand{\tabularPaddingBot}{\vspace{0.5cm}}

\newcommand{\tabitem}{~~\llap{\textbullet}~~}
\newcommand{\coloredHline}[1]{\arrayrulecolor{#1}\hline}
\newcommand{\minitab}[2][l]{\begin{tabular}{#1}#2\end{tabular}}

% quoting
\newcommand{\quoteSource}[2]{\glqq{#1}\grqq, \textit{#2}}
\newcommand{\quoteQuick}[1]{\glqq{#1}\grqq}
\newcommand{\quoteAuthor}[2]{\quoteQuick{#1}, von \textit{#2}}
\newcommand{\quoteCite}[2]{\quoteQuick{\textit{#1}}#2}
\newcommand{\srcSelf}{\\\text{Quelle: Eigene Darstellung}}
\newcommand{\citePage}[2]{\cite{#1} Seite: {#2}}
\newcommand{\citeEquation}[1]{\text{\hspace{1cm}{#1}}}

% paragraphs
\newcommand{\parTiny}{\vspace{2mm}\\}
\newcommand{\parSmall}{\\~\par}
\newcommand{\parNoIndent}{\\\noindent}
\newcommand{\parWrapfig}{\\\text{ }\\}

% math


% 
\newcommand{\emphFold}[1]{\textbf{#1}}

% page layout
\newcommand{\geometryNormal}{\newgeometry{top = 2cm, left=2.5cm,bottom=2cm, right = 4cm}}
\newcommand{\geometryToc}{\newgeometry{top = 3cm, left=3cm,bottom=3cm, right = 3cm}}

% todo notes
\newcommand{\todoLowp}[1]{\todo[color=green!40]{#1}}
\newcommand{\todoMedp}[1]{\todo[color=orange!40]{#1}}
\newcommand{\todoHighp}[1]{\todo[color=red!40]{#1}}
\newcommand{\todoFig}[1]{\missingfigure{#1}}

% titles
\colorlet{ctcolorchapternum}{black}

\newcommand\mysectionformatCore[1]{%
	\vspace{-3em}\parbox[b]{\dimexpr\textwidth-1em\relax}{\raggedright#1}}
\newcommand{\titleformatCore}
{
	\titleformat
	{\section}[display]%
	{\color{black}\Large\bfseries}%
	{\vspace{-18em}\raggedleft{%
			{\color{ctcolorchapternum}\fontsize{60}{60}{\selectfont\thesection}}%
		}\hspace{1em}%
	}%
	{0pt}%
	{\mysectionformatCore}%
	[{\vspace{-0.5em}%
		\rule{0.75\textwidth}{1pt}%
		\rule[-3pt]{0.25\textwidth}{4pt}}]
	
	\titleformat
	{\subsection}[block]%
	{\color{black}\large\bfseries}%
	{\thesubsection\text{ }}%
	{0pt}{}[]%
	
	\titleformat
	{\subsubsection}[block]%
	{\color{black}\normalsize\bfseries}%
	{\thesubsubsection\text{ }}%
	{0pt}{}[]%
	
	\titleformat{\paragraph}
	{\normalfont\normalsize\bfseries}{\theparagraph}{1em}{}
	\titlespacing*{\paragraph}
	{0pt}{3.25ex plus 1ex minus .2ex}{1.5ex plus .2ex}
}

\newcommand\mysectionformatAppendix[1]{#1}
\newcommand{\titleformatAppendix}
{
	\titleformat
	{\section}[display]%
	{\color{black}\Large\bfseries}%
	{}{0pt}{\mysectionformatAppendix}[]
}

% END NEWCOMMANDS

% BEGIN RENEW COMMANDS
% END RENEW COMMANDS

\newlistof{links}{lks}{List of Links}
\newcommand\externallink[1]{%
	\refstepcounter{links}%
	\footnote{\url{#1}}%
	\addcontentsline{lks}{links}{%
		\protect\numberline{\thelinks}%
		\protect\url{#1}}%
}
\newcommand\externallinkIL[1]{%
	\refstepcounter{links}%
	\addcontentsline{lks}{links}{%
		\protect\numberline{\thelinks}%
		\protect\url{#1}}%
}

\newcolumntype{L}[1]{>{\raggedleft\let\newline\\\arraybackslash\hspace{0pt}}m{#1}}
\newcolumntype{R}[1]{>{\raggedright\let\newline\\\arraybackslash\hspace{0pt}}m{#1}}
%\newcolumntype{C}[1]{>{\raggedcenter\let\newline\\\arraybackslash\hspace{0pt}}m{#1}}

\newcolumntype{C}[1]{>{\raggedcenter\hspace{0pt}}p{#1}}


\author{Matthias Elmar Gensheimer}
\title{EyeCandy3D Manual}
\date{\today}


% BEGIN HEADER
\newcommand{\headerNormal}
{
	\pagestyle{fancy}
	\fancyhf{}
	\lhead{\rightmark}
	\rhead{\thepage}
}

\newcommand{\headerEmpty}
{
	\fancyhf{}
	\renewcommand{\headrulewidth}{0pt}
	\lhead{}
	\rhead{}
}

\newcommand{\headerErklaerung}
{
	\pagestyle{fancy}
	\fancyhf{}
	\rhead{\thepage}
}
% END HEADER

% BEGIN FOOTER
\usepackage{scrlayer}
\DeclareNewLayer[
foreground,
foot,
hoffset=0pt,
width=\paperwidth,
contents={\parbox{\layerwidth}{\centering}}
]{PageMarkCentredToPage}
\RedeclarePageStyleByLayers{plain}{PageMarkCentredToPage}
% END FOOTER

% BEGIN TOC SETUP
\makeatletter
\patchcmd{\l@section}
{\hfil}
{\leaders\hbox{\normalfont$\m@th\mkern \@dotsep mu\hbox{.}\mkern \@dotsep mu$}\hfill}
{}{}
\makeatother
% END TOC SETUP

% BEGIN PAGE SETUP
\geometryNormal
\setlength{\extrarowheight}{3pt}
% END PAGE SETUP


\titleformatCore
\headerNormal

\begin{document}
	\pagenumbering{Roman} 
	
	\lstset{language=C++,
		keywordstyle=\color{blue},
		basicstyle=\ttfamily,
		commentstyle=\ttfamily\itshape\color{gray},
		stringstyle=\ttfamily,
		showstringspaces=false,
		breaklines=true,
		frameround=ffff,
		rulecolor=\color{black},
		frame=single,
		autogobble=true
	}
	
	\begin{titlepage}
		\begin{flushright}
			\includegraphics[width=0.5\textwidth]{resources/hs_kempten_logo}
		\end{flushright}
		\begin{center}
			\topskip0pt
			\vspace*{\fill}
					\LARGE\textbf{EyeCandy3D User Manual}\\		
		
			\LARGE{A 3D Scenegraph built on OpenGL}\\
			
			
			\vfill
			\Large{\textcopyright 2018 Matthias Gensheimer}\\
			\Large{All Rights Reserved.}
		\end{center}
		
	\end{titlepage}
	
	% Add an empty page
	\mbox{}
	\thispagestyle{empty}
	\newpage
	
	\pagebreak
	
	\geometryNormal
	\tableofcontents
	\geometryNormal
	\setstretch{1.30}
	
	\pagenumbering{arabic} 
	\section{Introduction}
	EyeCandy3D is a scene graph based on OpenGL. Every program consists of one \textbf{Application}, which holds a number of windows.
	Each window can hold a variable number of scenes, which can run simultaneously.
	
	
	
	\pagebreak
	\section{First steps}
		In this section I will guide you through the most important aspects of using this library.
		
		\subsection{Creating an application}
		An \textbf{Application} is the main part of this library.
		\begin{lstlisting}
			Application app;
		\end{lstlisting}
		It allows the creation of new windows, which will automatically visualized. The following function will create a new window and add it to the application:
		\begin{lstlisting}
			app.createWindow<WindowType>(...);
		\end{lstlisting}
		\textbf{WindowType} has to be an own \textbf{Window} implementation.
		
		\subsection{Creating a window}
		Custom windows must be derived from \textbf{ec::Window}. This window has to be instanced with the window creation function provided by an \textbf{Application}. 	
		\begin{lstlisting}
			class ExampleWindow : public ec::Window
			{
			public:
				explicit ExampleWindow(unsigned int windowWidth,
					unsigned int windowHeight,
					const std::string& title)
				// Own window implementation ...
			};
		
			ec::Application app;
			app.createWindow<ExampleWindow>(width, height, "Example Window", "id");
		\end{lstlisting}
		
		The dervied window  \textbf{must} support a constructor with the following signature:\\
		\textbf{MyWindow(unsigned int, unsigned int, const std::string\&)}
		
		
		
		\subsection{Creating a scene}
		Custom scenes must be derived from \textbf{ec::Scene}. By registering it with the scene manager in the associated window, it can receive updates.	
		
		\begin{lstlisting}
			class ExampleScene : public ec::Scene
			{
			public:
				explicit ExampleScene(const std::string& name, 
					ec::Window* window);
			
				// Own window implementation ...
			};
			
			// In ExampleWindow:
			void ExampleWindow::initScenes()
			{
				// Create new window
				auto exampleScene = new ExampleScene("example", this);
				
				// Register the new window, so it receives updates
				m_sceneSystem.registerScene(exampleScene);
			
				// ...
			}
		\end{lstlisting}
		
		\subsection{Resource registry}
		Built in types:
		\begin{itemize}
			\item 
		\end{itemize}
		
	\pagebreak
	\section{Scene graph}
		\subsection{Nodes}
			The scene graph is made out of \textbf{Node}s. The graph is directed and has to be acyclic. Each scene contains one root node, which will always exist and cannot be deleted (but it can be transformed, i.e. translated). The root node doesn't have a parent.
			
			Each node can have any number of children, but one father node at most. A node is a \textbf{Transform3D}, thus it can be translated, rotated and scaled. Besides a local matrix, it also contains a global matrix. Before each render cycle, its local and global matrices will be updated after the formula:
			\begin{algorithm}
				$\vec{x} := \vec{forward}$\\
				$\vec{y} := \vec{up}$\\
				$\vec{z} := \vec{x} x \vec{y}$\\
				$\vec{t} := \vec{pos}$\\
				$M_{local} := (x^T, y^T, z^T, t^T)$\\
				$M_{global} := M_{\texttt{global\_parent}} * M_{local}$\\
			\end{algorithm}
				
		\subsection{Texture}
			The texutre class encapsulates a texture, which resides on the gpu. There are two functions, which allow the creation of either 2D or 3D textures.			
			\begin{lstlisting}
				class Texture
				{
				public:
					explicit Texture();
					~Texture();
					
					/**
					* \brief Create a 2D texture.
					* \return True if creation was successful, false otherwise.
					*/
					bool textureFromFile(const char* path, const std::string& type);
					
					/** 
					* \brief Create a 3D texture.
					* \return True if creation was successful, false otherwise.
					*/
					bool cubeMapFromFile(const char* path, const std::string& type);
				};	
			\end{lstlisting}
		
		\subsection{Material}
			A \textbf{Material} contains a number of properties, which defines the color of an object. These properties include:
			\begin{itemize}
				\item Ambient, Diffuse, Specular colors
				\item Shininess factor for highlights
				\item Textures
			\end{itemize}
			\begin{lstlisting}
				
			\end{lstlisting}	
		
		\subsection{Geometry}
			A geometry object encapsulates various geometry data, including:
			\begin{itemize}
				\item Vertices
				\item Normals
				\item Texture coordinates
			\end{itemize}
			It is responsible for memory managment on the GPU regarding geometry data.
			
			There already are specific geometry implementations for common geometry types:
			\begin{description}
				\item[3D]: 
					\begin{packed_itemize}
						\item CubeGeometry
						\item SphereGeometry
						\item CylinderGeometry
					\end{packed_itemize}
				\item[2D]:
					\begin{packed_itemize}
						\item CircleGeometry
						\item RectangleGeometry
					\end{packed_itemize}
			\end{description}
			If these predefined types aren't sufficicent, you can create your own geometry types. This is done by deriving from \textbf{StaticGeometry}.
			
			
		\subsection{Shader}
			The \textbf{Shader} class can load, compile and link shader programs. The shading language used is GLSL, since OpenGL is being used.
		
		\subsection{Drawable}
			A drawable groups together
			\begin{packed_itemize}
				\item Geometry,
				\item Material and
				\item Shader
			\end{packed_itemize}
		
			Drawables can be added to scene graph \textbf{Node}s so they are rendered.
		
	\pagebreak
	\section{Rendering}
		
		\subsection{Frame}
		
		\subsection{Camera}
		
	
	\pagebreak
	\section{Lighting}
		TBA
		
		
		
	\pagebreak
	\section{Input}
	Input is provided through the GLFW library.
		
		\subsection{Input events}
		Input events are always bound to one window. There are multiple sources, which can generate input events:
		\begin{packed_itemize}
			\item \textbf{Mouse}
			\item \textbf{Keyboard}
			\item \textbf{Window}
			\item \textbf{Joystick}
		\end{packed_itemize}
		An \textbf{InputEvent} consists of an \textbf{InputType} and the \textbf{EventData}. The \textbf{EventData} holds all different kinds of events, of which only one can be active at a time due to it being a union. The \textbf{InputType} describes, which part of the \textbf{EventData} is active. All other elements inside \textbf{EventData} are invalid!
		
		I.e. if the \textbf{InputType} is \textit{key\_pressed}, only the KeyboardEvent inside the \textbf{EventData} is active.

		\subsection{EventSystem}
		An event system is always linked to exactly one window. 
		\begin{description}
			\item[DeviceRegistry:] \hfill\\
				Contains input devices (mouse, keyboard, joystick etc.). Those devices can be activated to generate input events.
			\item[InputObservable:] \hfill\\
				\textbf{InputListener} can be registered at this component, which will then be notified about incoming \textbf{InputEvents}.
			\item[InputListener:] \hfill\\
				Input listener contain a number of callbacks, which can be added by the user. Input listener have to be registered at an \textbf{InputObservable} located in a window, to be informed about input events.	
		\end{description}
		
			
	
	\pagebreak
	\section{GUI}
	The GUI is built using the Agui-library with custom backends for OpenGL.
	
		\subsection{Initialization}
		Before the GUI library-part can be used, it has to be initialized. To do so, use the static function:
		\begin{lstlisting}
			void ec::MiniAgui::init();
		\end{lstlisting}
		
		\subsection{Backend}
		
		\subsection{title}
		
		
	\pagebreak
	
	
	
	
	\pagenumbering{alph}
	\newpage
	\appendix
	\listoffigures
	\listoftables
	
	% TODO: glossary
	
\end{document}